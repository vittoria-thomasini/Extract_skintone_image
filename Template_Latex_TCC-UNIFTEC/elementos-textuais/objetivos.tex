\section{Justificativa}
\label{justificativa}

A motivação para o desenvolvimento do trabalho reside na problemática de muitos consumidores terem dificuldades em identificar o próprio tom de pele e conseguir identificar no mercado de cosméticos um produto que tenha um tom que combine com o seu. Atualmente, muito se fala de diversidade e inclusão na indústria de cosméticos e por isso cada vez mais marcas estão desenvolvendo cartelas com faixas de cores mais amplas.

No mercado de varejo entende-se que estratégias que auxiliam os clientes a executarem tarefas que entregam valor e reconhecimento pessoal são mais eficazes para fidelizar o cliente \cite{Snap_and_match_a_case_study_of_virtual_color_cosmetics_consultation}, por isso, pensando nisso e com a crescente demanda de personalização, com a tecnologia tornando-se uma ferramenta a fim de estimular a experiência dos consumidores. Surgiu a ideia do projeto de desenvolvimento utilizando técnicas de inteligência artificial para identificar e classificar o tom de pele para auxiliar os consumidores a identificarem o seu próprio tom de pele. 

Existem diversas abordagens para detecção de pele, mas poucos estudos que se dedicam a estudar tons de pele em si. Maior parte dos estudos se dedicam a averiguar melhores abordagens para detecção de pele voltada ao reconhecimento facial. Por isso, para o desenvolvimento do trabalho buscou se focar na identificação com menos interferências possíveis de iluminação e fundo, assim como a influência a região dos olhos, boca, nariz e cabelos. Por isso precisou-se de uma avaliação e comparação entre técnicas a serem utilizadas. Este trabalho foi realizado para realizar a avaliação e de propor uma aplicação de detecção de tom de pele que atenda os seguintes requisitos: precisão e desempenho.

O requisito de precisão para ser possível a utilização de fotos sem ambiente controlado de iluminação e qualidade de imagem. Dessa forma, o sistema deverá detectar o tom de pele humana em imagens com iluminações inadequadas, com o mínimo de interferência de fundo e com variadas resoluções. Além disso, o requisito de desempenho para que o sistema que utilize o programa de identificação e classificação processe as imagens em tempo real, sendo possível o cliente tirar a própria foto e já possuir o resultado da avaliação.

Além disso, diversas aplicações poderão usar os resultados da detecção e classificação de tom de pele, principalmente voltado a indústria da beleza, por profissionais da área de visagismo, maquiadores, pesquisadores sociais, entre outros.

\chapter{Objetivos}
Com o propósito de identificar subtons de pele utilizando técnicas de inteligência artificial, foram estabelecidos alguns objetivos descritos nesta seção. A Seção~\ref{objetivosgerais} apresenta o objetivo principal do trabalho, enquanto a Seção~\ref{objetivosespecificos} detalha os objetivos específicos e por fim a seção~\ref{justificativa} apresenta as justificativas que motivaram o desenvolvimento deste estudo.

\section{Objetivos Gerais}
\label{objetivosgerais}

O objetivo principal deste trabalho é detectar e identificar tons e subtons de pele humana em imagens, utilizando a escala \textit{Monk Skin Tone}(CITAÇÃO) como métrica de classificação, com o intuito de sugerir melhores formulações de cartela de cores para futuras aplicações de recomendação de produtos de maquiagem.
\section{Objetivos Específicos}
\label{objetivosespecificos}

Para o objetivo principal poder ser obtido, os seguintes objetivos específicos foram propostos:

\begin{itemize}
    \item[--] Realizar uma pesquisa bibliográfica sobre o tema proposto;
    \item[--] Identificar a técnica adequada de\textsl{Machine Learning} a ser aplicada;
    \item[--] Obter e tratar as imagens utilizando a biblioteca de manipulação de imagens OpenCV;
    \item[--] Segmentar as imagens para extrair informações da imagem relevantes para classificação de tons da pele; 
    \item[--] Desenvolver algoritmos de identificação da escala de tons a partir da extração da cor dominante da pele;
    \item[--] Classificar o tom dominante identificado conforme a paleta de Monk; 
    \item[--] Validar a classificação do tom dominante identificado conforme a paleta de Monk; 
\end{itemize}