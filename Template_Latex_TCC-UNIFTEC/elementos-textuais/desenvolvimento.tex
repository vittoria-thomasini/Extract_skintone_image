\chapter{Desenvolvimento}
\label{cap:desenvolvimento}

Este capítulo apresenta o processo de desenvolvimento do código e seu funcionamento, o qual é responsável pela identificação de tons de pele dos indivíduos nas imagens estáticas com detalhes os quais obteve-se o resultado esperado. 

Para a detecção de pele não se faz necessário a escolha de apenas uma região corporal para a sua identificação.Contudo, como objetivo do trabalho a detecção de face é o primeiro passa para a resolução do problema. O objetivo da deteção da face é encontrar rostos a partir de imagens e para isso foi utilizado o algoritmo haas cascade para a detecção.
A função de detecção facial tem como objetivo localizar a partir de qualquer imagem, inicialmente, com rosto em angulo frontal, e recortar a imagem com relação ao tamanho da face humana apresentada. O diaframa desta etapa é apresentado na Figura.


