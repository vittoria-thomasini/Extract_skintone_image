\chapter{Metodologia}
\label{cap:metodologia}

\subsection{Revisão Sistemática da Literatura}
 Nesse capítulo é apresentada a metodologia de Revisão Sistemática da Literatura utilizada para o rastreio de trabalhos relacionados. Este, o qual, possui o intuito de pesquisar artigos e trabalhos relacionados e aprofundar os conhecimentos sobre o tema proposto, com objetivo de entender as abordagens utilizadas no meio científico e os desafios a fim de aperfeiçoar o desenvolvimento deste projeto. Conforme essa metodologia, foram seguidas as seguintes etapas:
 
Na primeira etapa, buscou-se por artigos e trabalhos relacionados nas bases digitais de pesquisa Google Acadêmico\footnote{https://scholar.google.com/}, IEEE Xplore\footnote{https://ieeexplore.ieee.org/Xplore/home.jsp} e Researchgate\footnote{https://www.researchgate.net/}. Nelas foram aplicadas sentenças de busca por palavras-chave que trouxessem trabalhos relacionados ao tema proposto. 

Organizado por relevância acadêmica e por pesquisa avançada, filtrando-se as pesquisas com as palavras-chave alcançou-se a seguinte expressão:

\begin{center}
\centering
\textbf{(("All Metadata":skin color) OR ("All Metadata":skin undertone) OR ("All Metadata":skin tone) OR ("All Metadata":artificial inteligence) OR ("All Metadata":neural network) OR ("All Metadata":diversity) OR ("All Metadata": computer vision) OR ("All Metadata":inclusion) OR ("All Metadata":beauty industry) OR ("All Metadata":cosmetics industry) OR ("All Metadata":skin color detection) NOT ("All Metadata":texture) NOT ("All Metadata":lesion) NOT ("All Metadata":medical) NOT ("All Metadata":cancer) NOT ("All Metadata":melanome))}
\end{center}


Na segunda etapa, foi realizado a triagem dos artigos anteriormente encontrados onde se buscou por artigos publicados entre os anos de 2018 e 2023. Em seguida, na terceira etapa foi realizada a leitura utilizando os títulos e resumos como critério de seleção. Nessa etapa foi visto que a maioria dos artigos selecionados anteriormente tinham como foco principal identificar pele em imagens e vídeos sendo a identificação da cor de pele somente uma etapa do processo e sem classificação das mesmas. Na quarta etapa foi feito a leitura completa dos artigos e buscou-se verificar a adequabilidade dos artigos selecionados. 

Como critério de análise dos artigos de inclusão levou-se em conta a abordagem dos estudos relacionados ao desenvolvimento de cosméticos e abordagens gerais utilizando \textit{Machine Learning}. Como critério de exclusão optou-se por não considerar artigos relacionados a problemas de saúde na pele como câncer de pele e feridas, apesar de ser um tópico bastante estudado utilizando técnica de inteligência artificial para identificação coloração de pele, segmentação e classificação. Como já é conhecido que a maioria das pesquisas realizadas sobre a análise da cor da pele é focado na detecção de pele\cite{A_survey_of_skin-color_modeling_and_detection_methods}. Como resultado foram encontrados no total 69 artigos, analisados os resumos passaram-se a 35 artigos e após as quatro etapas do RSL restaram somente 4 trabalhos relacionados.

A metodologia deste estudo consistiu em quatro
etapas principais, ou seja, detecção de face, amostragem de pele por imagem, extração de pixel de pele e classificação de pixel de pele. Ao
longo deste artigo, nosso método é uma classificação baseada em
pixels, onde os pixels são a principal característica



Detecção de cor de pele é um processo para encontrar pixels com tons de pele em regiões definidas a partir de uma imagem. 

\subsection{Materiais e Métodos}
Para os experimentos abordados no desenvolvimento foi utilizado a linguagem Python\footnote{https://www.python.org/doc/essays/blurb/}. A escolha da linguagem se deu pela facilidade de uso e pela quantidade de estudos na área de \textit{Machine Learning} que utilizam a mesma.

LEITURA DE DATASET STILL UNDEFINED.ATUALMENTE UTILIZANDO BIBLIOTECA IMUTILS PARA LER UMA UNICA IMAGEM POR VEZ.





As funções que implementam as validações de modelo como o cálculo de precisão, acurácia, margem de erro foram construídos utilizando a biblioteca Scikit-learn\footnote{https://scikit-learn.org/stable/about.html} . 




(deve aparecer os objetivos)

(como validar?)