\chapter{Metodologia}
\label{cap:metodologia}
Neste capítulo são apresentadas as metodologias aplicadas para o desenvolvimento dos objetivos descritos no Capítulo \ref{cap:introducao}, além dos materiais utilizados durante o processo.

\section{Revisão Sistemática da Literatura}
Com o objetivo de aprofundar os conhecimentos sobre o tema proposto, a fim de compreender as abordagens utilizadas na área e os desafios elencados neste projeto, realizou-se uma Revisão Sistemática da Literatura. As etapas observadas foram as seguintes:

%Nesse capítulo é apresentada a metodologia de Revisão Sistemática da Literatura utilizada para o rastreio de trabalhos relacionados. Este, o qual, possui o intuito de pesquisar artigos e trabalhos relacionados e aprofundar os conhecimentos sobre o tema proposto, com objetivo de entender as abordagens utilizadas no meio científico e os desafios a fim de aperfeiçoar o desenvolvimento deste projeto. Conforme essa metodologia, foram seguidas as seguintes etapas:
 
Na primeira etapa, buscou-se por ,artigos e trabalhos relacionados nas bases digitais de pesquisa Google Acadêmico\footnote{https://scholar.google.com/}, IEEE Xplore\footnote{https://ieeexplore.ieee.org/Xplore/home.jsp} e Researchgate\footnote{https://www.researchgate.net/}. Nelas foram aplicadas sentenças de busca por palavras-chave que trouxessem trabalhos relacionados ao tema proposto. 

Organizado por relevância acadêmica e por pesquisa avançada, filtrando-se as pesquisas com as palavras-chave alcançou-se a seguinte expressão:

\begin{center}
\centering
\textbf{(("All Metadata":skin color) OR ("All Metadata":skin undertone) OR ("All Metadata":skin tone) OR ("All Metadata":artificial inteligence) OR ("All Metadata":neural network) OR ("All Metadata":diversity) OR ("All Metadata": computer vision) OR ("All Metadata":inclusion) OR ("All Metadata":beauty industry) OR ("All Metadata":cosmetics industry) OR ("All Metadata":skin color detection) NOT ("All Metadata":texture) NOT ("All Metadata":lesion) NOT ("All Metadata":medical) NOT ("All Metadata":cancer) NOT ("All Metadata":melanome))}
\end{center}

Na segunda etapa, foi realizado a triagem dos artigos anteriormente encontrados, onde se buscou por artigos publicados entre os anos de 2018 e 2023. Em seguida, na terceira etapa foi realizada a leitura utilizando os títulos e resumos como critério de seleção. Nessa etapa foi visto que a maioria dos artigos selecionados anteriormente tinham como foco principal identificar pele em imagens e em vídeos, sendo a identificação da cor de pele somente uma etapa do processo e sem classificação das mesmas. Na quarta etapa foi feita a leitura completa dos artigos e buscou-se verificar a adequabilidade dos artigos selecionados. 

Como critério de análise dos artigos de inclusão levou-se em conta a abordagem dos estudos relacionados ao desenvolvimento de cosméticos e abordagens gerais utilizando \textit{Machine Learning}. Como critério de exclusão optou-se por não considerar artigos relacionados a problemas de saúde na pele como câncer de pele e feridas, apesar de ser um tópico bastante estudado utilizando técnica de inteligência artificial para identificação, coloração de pele, segmentação e classificação. Como já é conhecido que a maioria das pesquisas realizadas sobre a análise da cor da pele é focado na detecção de pele \cite{A_survey_of_skin-color_modeling_and_detection_methods}. Como resultado foram encontrados no total 69 artigos, analisados os resumos passaram-se a 35 artigos e após as quatro etapas do RSL restaram somente 4 trabalhos relacionados.

\subsection{Definição de etapas de estudo}
Para identificar as etapas necessárias e os algoritmos para construir o processo e montar a arquitetura utilizou-se a pesquisa bibliográfica. Para o desenvolvimento dos algoritmos, utilizou-se a pesquisa experimental de caráter exploratório. Assim, a metodologia deste estudo consistiu nos pilares relatados no Capítulo \ref{cap:trabalhos-relacionados} sendo eles coletar diferentes imagens de pele, tratar imagens considerando problema relacionados à iluminação, escolher sistemas de cores adequados para conversão das imagens e excluir e separar de regiões de pele e não pele nas imagens, além de identificar a cor dominante e classificar o tom de pele.

Para os experimentos abordados no Capítulo \ref{cap:desenvolvimento} foi utilizada a linguagem Python\footnote{https://www.python.org/doc/essays/blurb/}, pela facilidade de uso e pela quantidade de bibliotecas e funcionalidades na área de \textit{Machine Learning} disponíveis. O ambiente utilizado para identificar e classificar tons de pele foi a plataforma Google Colab \footnote{https://colab.google/} e Visual Code Studio. E, para o versionamento e armazenamento foi utilizado o Git e o Github.

Para a avaliação do resultado de classificação das identificações de tom dominante optou-se por utilizar um \textit{dataset} como referência. Para a escolha do \textit{dataset} foi avaliado a região de coleta de dados, quantidade uniforme de tons de pele e a qualidade da fonte de dados. Priorizou-se a coleta de dado de imagens de rostos de pessoas brasileiras na posição frontal com diferentes condições de iluminação. Para isso foram analisados 14 \textit{datasets} com variações de fotos de pessoas, onde apenas 2 continham fotos de pessoas brasileiras, a base de dados FEI Face \footnote{https://fei.edu.br/~cet/facedatabase.html} desenvolvido pelo Laboratório de Robótica - FEI e a MST-E do \textit{Projeto Skintone} do \textit{Google Research}. A fim de facilitar a classificação optou-se por utilizar o \textit{dataset} MSTE \footnote{https://skintone.google/mste-dataset}, já que as imagens já estavam catalogadas conforme a escala Monk pelo próprio DR. Ellis Monk, não requerendo outro especialista para classificar em primeira instância. 

O \textit{dataset} \textit{Monk Skin Tone} MST-E foi desenvolvido pelo \textit{Google Skin Tone} e contém exemplos de 19 pessoas que abrangem a escala de Monk. A base contém 1515 imagens e 31 vídeos. Cada pessoa foi fotografada em diversas poses, com acessórios e com diversas condições de iluminações. Para o trabalho apenas são utilizadas as imagens no qual as pessoas estão de frente com boa e má iluminação para avaliar se a classificação do programa desenvolvido está classificando conforme esperado.
