\chapter{Introdução}
\label{cap:introducao}

A indústria de cosméticos do Brasil é uma das maiores do mercado de cosméticos mundial, sendo em 2022 responsável por exportar mais de 770 milhões de dólares em produtos de beleza e higiene\cite{Cosmetics_industry_in_Brazil}. A maquiagem além de uma arte corporal que já é utilizada a mais de 7000 anos também tem impacto pessoal e social a vida de quem o usa e das pessoas ao redor. Muitas vezes, a maquiagem é usada na intenção de realçar a beleza, onde pode gerar um efeito sobre a atratividade facial, corrigindo imperfeições e realçando características. Além disso, pode estar associada com maiores atribuições de sucesso profissional, renda salarial e posição socioeconômico \cite{Respostas_Emocionais_Implícitas_Julgamento_da_Atratividade_Facial_em_Faces}.

 Nessa indústria, a cor é uma característica chave para os consumidores e  fabricantes de cosméticos. E, entre os cosméticos disponíveis no mercado, encontrar a base que combina com a cor da pele pode ser decisivo para um bom acabamento. Contudo, diferentes marcas desenvolvem diferentes cartelas de cores, para tipos de pele diferentes e tipos acabamentos, resultando em inúmeras opções de bases disponíveis no mercado. Isso sem considerar que as marcas no segmento de beleza podem ser categorizada de por tipos diferentes de produtos como sendo naturais, urbano, jovem, orgânicos, veganos, \textit{cruelty free}, limpos e entre outros. Por isso, embora haja opções disponíveis, também há a dificuldade do cliente em encontrar o tom que o atenda. Principalmente em um país de grande miscigenação como Brasil, onde podemos encontrar todos os fotótipos Fitzpatrick espelhados pelo país e múltiplos “subfototipos”.\cite{Régua_de_Pele_Linha_de_Maquiagem_para_a_Mulher_Brasileira}

Segundo o artigo \cite{Snap_and_match_a_case_study_of_virtual_color_cosmetics_consultation} foi conduzido pela empresa Estée Lauder um estudo que aponta que 70\% das mulheres não conseguem encontrar o tom exato que combina com a cor da pele e que 94\% das mulheres estão usando a cor errada da base. Isso pode ocorrer porque muitas mulheres brasileiras por serem muito brancas categorizadas como fototipo I, morenas escuras como fototipo V, e  negras, fototipo VI\cite{Régua_de_Pele_Linha_de_Maquiagem_para_a_Mulher_Brasileira}.
A maneira tradicional do consumidor de determinar qual a base de cor combina com a pele requer aplicar tons disponíveis na loja na pele \cite{A_development_of_a_portable_device_for_skin_color_estimation_on_cosmetic_foundation_applying} ou por seleção visual, sendo a compra também em loja virtual ou com a ajuda de vendedores em uma loja \cite{Snap_and_match_a_case_study_of_virtual_color_cosmetics_consultation}.
E, no mercado profissional, os maquiadores misturam dois ou mais tons de base para encontrar a cor ideal da cliente. Contudo, dependendo das condições de iluminação, pode levar a uma aparência não natural comparado a outra condição de iluminação.  Isso pode ser causado principalmente por problemas ópticos, já que espectros da pele podem ser bem diferentes dos das bases cosméticas. Assim, como a escolha da região de teste pode ter diferença de coloração com a região de uso, como, por exemplo, testar no braço ou pescoço o tom, sendo que o produto será utilizado para uniformizar a coloração facial. Nesse caso, a ajuda ao consumidor na hora da escolha pode ser uma estratégia para fidelizar o cliente.
Nesse sentido, empresas de cosméticos apostam em serviços personalizados usando principalmente da realidade virtual, a realidade aumentada e a consultoria para auxiliar o consumidor na escolha de seus produtos.


